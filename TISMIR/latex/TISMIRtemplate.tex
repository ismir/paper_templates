% %%%%%%%%%%%%%%%%%%%%%%%%%%%%%%%%%%%%%%%%%%%%%%%%%%%%%%%%%%%%%%%%%%%%%%%%%%%%%%%%
% Template for TISMIR Papers
% 2017 version, based on previous ISMIR conference template
% %%%%%%%%%%%%%%%%%%%%%%%%%%%%%%%%%%%%%%%%%%%%%%%%%%%%%%%%%%%%%%%%%%%%%%%%%%%%%%%%

\documentclass{article}
\usepackage[utf8]{inputenc}
\usepackage{tismir,amsmath,url}
\usepackage{graphicx}
\usepackage{booktabs}

% %%%%%%%%%%%%%%%%%%%%%%%%%%%%%%%%%%%%%%%%%%%%%%%%%%%%%%%%%%%%%%%%%%%%%%%%%%%%%%%%
% Title and Author information
% %%%%%%%%%%%%%%%%%%%%%%%%%%%%%%%%%%%%%%%%%%%%%%%%%%%%%%%%%%%%%%%%%%%%%%%%%%%%%%%%

\title{Paper Template For TISMIR}
%
\author{First Author1\thanks{University One, Campus Rd. 3, 12345 Mirland},%
~Second Author2\thanks{Research Institute Two, Campus Rd. 4, 12345 Flatland},%
~and Third Author3\footnotemark[1]}

\date{}

% %%%%%%%%%%%%%%%%%%%%%%%%%%%%%%%%%%%%%%%%%%%%%%%%%%%%%%%%%%%%%%%%%%%%%%%%%%%%%%%%
% Setup for Page Headers
% %%%%%%%%%%%%%%%%%%%%%%%%%%%%%%%%%%%%%%%%%%%%%%%%%%%%%%%%%%%%%%%%%%%%%%%%%%%%%%%%

\usepackage{fancyhdr}

% adjust pagestyle "plain" which is used on the first page
\fancypagestyle{plain}{%
  \fancyhf{}
  \renewcommand{\headrulewidth}{0.6pt}

  \fancyhead[L]{%
    \begin{minipage}{0.35\textwidth}
    \sf \small Transactions of the International
    Society for Music Information Retrieval
  \end{minipage}
  \hfill
  \begin{minipage}{0.6\textwidth} \sf \small
    Author1, F., Author2, S. and Author3, T. (2017).
    Paper Template for TISMIR,
    \textit{Transactions of the International Society for
    Music Information Retrieval}, V(N),
    pp.\ xx--xx, DOI: https://doi.org/xx.xxxx/xxxx.xx
  \end{minipage}
  \vspace{0.5em}
  }
}

% pagestyle "fancy" is used in the rest of the document
\pagestyle{fancy}
\renewcommand{\headrulewidth}{0pt}
\fancyhf{}
\lhead{\sf \thepage}
\rhead{\sf Author1, Author2, Author3: Paper Template For TISMIR}

\sloppy % please retain sloppy command for improved formatting

%%%%%%%%%%%%%%%%%%%%%%%%%%%%%%%%%%%%%%%%%%%%%%%%%%%%%%%%%%%%%%%%%%%%%%%%%%%%%%%%
% Document Content
%%%%%%%%%%%%%%%%%%%%%%%%%%%%%%%%%%%%%%%%%%%%%%%%%%%%%%%%%%%%%%%%%%%%%%%%%%%%%%%%

\begin{document}

%%%%%%%%%%%%%%%%%%%%%%%%%%%%%%
% Abstract
%%%%%%%%%%%%%%%%%%%%%%%%%%%%%%

\twocolumn[{%
\textsf{\bfseries \large ARTICLE TYPE}
\maketitle
%
\rule{\textwidth}{0.4pt}
%
\begin{abstract}
\noindent
% Edit abstract from here ------------------------->
Research articles must have the main text prefaced by an abstract of no more
than 250 words summarising the main arguments and conclusions of the article.
This must have the heading ``Abstract'' and be easily identified from the start
of the main text.
% <---------------------------------------- to here
\end{abstract}
%
\begin{keywords}
Up to six keywords (optional).
\end{keywords}
}]
\saythanks{}

%%%%%%%%%%%%%%%%%%%%%%%%%%%%%%
% Main Content Start
%%%%%%%%%%%%%%%%%%%%%%%%%%%%%%

\section{Headings}\label{sec:headings}

Up to three level headings may be present and must be clearly identifiable
using different font sizes, bold or italics. IF accepted for publication,
these will be converted into journal during typesetting.

As a general rule, submissions should be structured so that introduction,
methods, conclusions and discussion are all clearly indicated by a first level heading.

\subsection{Level heading 2}

Lorem ipsum dolor sit amet, consectetur adipiscing elit.
Maecenas finibus posuere lorem nec pulvinar. Donec varius
rhoncus leo nec semper. Nulla pulvinar ex nulla, sit amet feugiat
nisi laoreet auctor. Cras tincidunt, neque non lacinia porttitor,
libero ante placerat justo, ut iaculis ante libero quis justo.
Ut sed volutpat leo. Curabitur
Etiam nunc elit, venenatis et gravida non, commodo sed massa.
Sed cursus, massa ornare fringilla rhoncus, nisi diam consectetur urna,
id fringilla quam orci at magna.

\subsubsection{Level heading 3}

Lorem ipsum dolor sit amet, consectetur adipiscing elit.
Maecenas finibus posuere lorem nec pulvinar. Donec varius rhoncus leo
nec semper. Nulla pulvinar ex nulla, sit amet feugiat nisi laoreet auctor.
Cras tincidunt, neque non lacinia porttitor, libero ante placerat justo,
ut iaculis ante libero quis justo. Ut sed volutpat leo. Curabitur ligula
lorem, tempus eu ipsum non, tempus volutpat ipsum. Nulla at fermentum ex.
Nullam ultrices et enim et venenatis. Nam maximus id risus pharetra ultrices.
Praesent auctor porta dolor at accumsan.
Integer justo metus, iaculis vitae egestas pulvinar, rhoncus nec ipsum.
Sed vel porttitor libero.

\section{Footnotes}\label{sec:footnotes}

Use endnotes rather than footnotes
(we refer to these as ``Notes'' in the online publication).
These will appear at the end of the main text, before ``References''.
All notes should be used only where crucial clarifying information
needs to be conveyed.
Avoid using notes for purposes of referencing, with in-text citations used
instead.
If in-text citations cannot be used, a source can be cited as part of a note.
Please insert the endnote marker after the end punctuation.

\section{Figures}\label{sec:figures}

Figures must all be cited in the main text, in sequential order.
All figures should be placed within the text file upon submission and during
the review process. Figures/images have a resolution of at least 150 dpi
(300 dpi or above preferred). The files are in one of the following formats:
JPG, TIFF, GIF, PNG, EPS (to maximise quality,
the original source file is preferred).

After editorial acceptance, you will be asked to provide the figure
images in high resolution files, rather than in the submission file,
so that this quality is transferred into the publication.
The typesetting process will place the figures in an appropriate
location in the PDF.

Each figure must have an accompanying descriptive main title.
This should clearly and concisely summarise the content and/or
use of the figure image.
A short additional figure legend is optional to offer a further description.

\begin{figure}[htbp]
  \centering
  \includegraphics[width=0.9\columnwidth]{figure}
  \caption{Figure captions should be placed below the figure.}
\label{fig:figure}
\end{figure}

\section{Tables}\label{sec:tables}

Tables must be created using a word processor's table function,
not tabbed text.
Tables should be included in the manuscript.
The final layout will place the tables as close to their first
citation as possible.

All tables must be cited within the main text, numbered with Arabic
numerals in consecutive order (e.g. Table 1, Table 2, etc.).

Each table must have an accompanying descriptive title.
This should clearly and concisely summarise the content and/or
use of the table.
A short additional table legend is optional to offer a further
description of the table.

\subsection{Tables should not include}

\begin{itemize}
  \item Rotated text
  \item Colour to denote meaning (it will not display the same on all devices)
  \item Images
  \item Vertical or diagonal lines
  \item Multiple parts (e.g. ``Table 1a'' and ``Table 1b'').
  These should either be merged into one table,
  or separated into ``Table 1'' and ``Table 2''.
\end{itemize}

\begin{table}[htpb]
\centering
  \begin{tabular}{ll}
  \toprule
  \bfseries String Value & \bfseries Numeric Value \\ \midrule
  Hello ISMIR  & 2017          \\
  \bottomrule
  \end{tabular}
  \caption{Table captions should be placed below the table.}
\label{tab:table}
\end{table}

\section{Equations}\label{sec:equations}

Equations should be placed on separate lines and numbered.
The number should be on the right side, in parentheses,
as in Eqn.~(\ref{eq:eq}).

\begin{align}\label{eq:eq}
E = mc^2
\end{align}

\section{Reproducibility (if applicable)}

If the content of your submission relates to data or software
that has been deposited in a code or preservation repository,
please provide summary information here, along with a DOI that
links to the deposited code/data.

\section{Acknowledgements}

Any acknowledgements must be headed and in a separate paragraph,
placed after the main text but before the reference list.

\section{Competing interests}

If any of the authors have any competing interests then these
must be declared. A short paragraph should be placed before
the references.
Guidelines for competing interests are available online.%
\endnote{Link to guidelines: \\
  \url{http://transactions.ismir.net/information/competingInterestGuidelines}}

\section{Ethics and consent}

If your research human or animal subjects, the appropriate ethics
committee must have approved the research and a statement added here,
including the name of the committee and reference number of approval.
Human subjects must have provided informed prior consent
to participate in the study.

\section{References}

All citations must be listed at the end of the text file,
in alphabetical order of authors' surnames.
References should not be listed if they are not cited in
the main text.
This journal uses the APA system.
Please visit the journal's website
for examples of referencing format.\endnote{Link to citation examples: \\
\url{http://transactions.ismir.net/about/submissions/\#References}}

In this template, you can use \verb=\citep{}= to include references
surrounded by parentheses, such as~\citep{KneesS16_MusicSimilarityRetrieval_SPRINGER}, and \verb=\cite{}=
for references embedded in the text,
such as~\cite{WeihsJVR16_MusicDataAnalysis_CRC},
\cite{SerraEtAl13_RoadmapMIR_CreativeCommon},
or~\cite{Mueller15_FMP_SPRINGER}.


%%%%%%%%%%%%%%%%%%%%%%%%%%%%%%%%%%%%%%%%%%%%%%%%%%%%%%%%%%%%%%%%%%%%%%%%%%%%%%%%
% Please do not touch.
% Print Endnotes
\theendnotes{}
%%%%%%%%%%%%%%%%%%%%%%%%%%%%%%%%%%%%%%%%%%%%%%%%%%%%%%%%%%%%%%%%%%%%%%%%%%%%%%%%

\section*{Acknowledgements}

The authors would like to thank\dots

%%%%%%%%%%%%%%%%%%%%%%%%%%%%%%%%%%%%%%%%%%%%%%%%%%%%%%%%%%%%%%%%%%%%%%%%%%%%%%%%
% Bibliography
%%%%%%%%%%%%%%%%%%%%%%%%%%%%%%%%%%%%%%%%%%%%%%%%%%%%%%%%%%%%%%%%%%%%%%%%%%%%%%%%

% For bibtex users:
\bibliography{TISMIRtemplate}

% For non bibtex users:
%\begin{thebibliography}{citations}
%
%\bibitem {Author:00}
%E. Author.
%``The Title of the Conference Paper,''
%{\it Proceedings of the International Symposium
%on Music Information Retrieval}, pp.~000--111, 2000.
%
%\bibitem{Someone:10}
%A. Someone, B. Someone, and C. Someone.
%``The Title of the Journal Paper,''
%{\it Journal of New Music Research},
%Vol.~A, No.~B, pp.~111--222, 2010.
%
%\bibitem{Someone:04} X. Someone and Y. Someone. {\it Title of the Book},
%    Editorial Acme, Porto, 2012.
%
%\end{thebibliography}

\end{document}
